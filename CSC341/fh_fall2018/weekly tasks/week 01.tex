\documentclass[11pt, oneside]{article}   	% use "amsart" instead of "article" for AMSLaTeX format
\usepackage{geometry}                		% See geometry.pdf to learn the layout options. There are lots.
\geometry{letterpaper}                   		% ... or a4paper or a5paper or ... 
%\geometry{landscape}                		% Activate for rotated page geometry
%\usepackage[parfill]{parskip}    		% Activate to begin paragraphs with an empty line rather than an indent
\usepackage{graphicx}				% Use pdf, png, jpg, or eps§ with pdflatex; use eps in DVI mode
								% TeX will automatically convert eps --> pdf in pdflatex		
\usepackage{amssymb}
\usepackage{amsmath}
\usepackage{enumitem}
%SetFonts

%SetFonts


\title{CSC 341: Automata, Formal Languages, and Complexity Theory}
\author{Fahmida Hamid}
%\date{}							% Activate to display a given date or no date

\begin{document}
\maketitle
\section{Day 01: Introduction}
\subsection{Welcome Speech!}
Greet the students!
Introduce myself, spend 5 mins talking about myself. Tell them about your office hours and how to reach you or schedule an appointment. Then introduce the mentor (David) and the grader (Faizaan). Now talk a little bit about the course policies. Highlight the exams dates, talk about the homework and submission policy, the p-web, collaboration policy, etc.
\begin{itemize}
\item Exams
\begin{enumerate}
\item Hour-Exam 01 (October 05 Friday)
\item Hour-Exam 02 (November 05 Monday)
\item Final Exam (December 20 Thursday)
\end{enumerate}
\item Homework: 8 individual homework, 1 group class presentation
\end{itemize}
Now start talking about the course: {\bf This is a traditional theory course in CS!} So, what do we cover in this course? Why is it important? Let me start with some examples. 

\subsection{Reduce one problem to another}
We try to solve a lot of problems every day: we define algorithms; based on them we write programs; we solve mathematical equations to establish the logical correctness of our statements or hypothesis, and so on. Sometimes we succeed, sometimes we fail. Sometimes we use the same technique	 to solve two apparently different problems. I am sure you have experienced these situations. That said, I would like to say that {\bf there are a lot of subtle connections between problems.} 
\par Let's say that we have only 2 problems ($A$ and $B$). One has solved problem $B$. No one has been able to solve problem $A$. One day you have discovered that you can model problem $A$ as problem $B$. Great! Now you know the solution to problem $B$ as well as $A$. Once you get the solution of $B$, you just have to convert it back to the parameters of $A$. 
Here are the steps:
\begin{enumerate}[noitemsep, nolistsep]
\item Model $A$ as $B$
\item Solve $B$ and produce result $R$
\item Convert $R$ using the parameters of $A$
\end{enumerate}

\subsubsection{Illustration}
{\bf Example:} I get an email from the department chair that we have a meeting at 9:00 AM. It is 8:45 AM. I How quickly can I go to my office? (provided, there are multiple possible routes and I am new in the town).  
(AKA) finding shortest path\\

{\bf Example:} I am asked to schedule the final exams so that students do not have time conflicts!
(AKA) finding maximum independent sets. Literally, give a small problem.\\
{\bf Spend some time on the open discussion!}\\

{\bf Point: One just has to know how to convert one problem to another!}

\subsection{Problems: Level of Complexity/Difficulty}

Let us look back at the two problems we just discussed. Are they equally difficult? Which one looks more difficult?
Why do you think that problem $XYZ$ is more difficult than problem $ABC$? We should also know what we mean by {complexity}.\\
{\bf Spend some time on the open discussion!}

\subsection{Proofs: why writing proofs are important?}

Once you develop a solution and it looks like that it is working for a set of examples that you have right now, are you confident that it will always succeed? 
\par If you know the solution to a problem, how do you explain it to others? Will they take your words for granted? How will you convince them that it works under all circumstances that may arise? How do you convince others that your solution will work for this problem of size $n = 1, 2, 3, \dots, n\in N$? Is your approach the best or most efficient one? Can you improve it further?

\subsection{Intractable and Unsolvable Problems}
With the advent of science, have we discovered solutions to all the problems? Can we think of a problem that we don't know how to solve? Or whether there is a universal solution or not? Sometimes the problem sounds very simple but we don't know how to solve it efficiently. 

{\bf Example: talk about circuit satisfiability}
 
\subsection{Course Overview}
In this course, we focus on three traditionally central area of the theory of computation: automata, computability, and complexity.

\subsubsection{Goals and Objectives}
There is no chronological order in which we will attain our goals but here is what we think we are going to practice in the course.
\begin{enumerate}[noitemsep, nolistsep]
\item to understand a problem and it's corner cases.
\item to learn and practice how to write formal proofs.
\item to understand when a problem is intractable or undecidable
\end{enumerate}


\subsection{Proof Techniques}
Give the hand-out (A proof written in 6 possible ways) to the students. Tell them that we will go over it in the next class.

\section{Day 02: Proof Techniques}

\subsection{Looking Back!}
I have a few questions for you:
\begin{itemize}[noitemsep, nolistsep]
\item In few sentences state the objectives and goals of this class.
\item Name the central area of the theory of computation. Define each area in one/two sentence(s).
\item Think of a real life problem (A) that can be reduced to another problem (B) that you have learnt in the some of the previous classes.
\item We have two problems in hand: sort the list of enrolled students in alphabetic order of their names and from an undirected graph, find the largest clique? Are both of the problems equally difficult? Why/why not?
\end{itemize}
Return the copy to your instructor with you and your partner's name on it.

\subsection{Proof Techniques}
Spend 5 minutes on Proof 1 from Professor Henry's note. How do you feel about it? Is it very easy to understand? Explain the proof to your friend.\\
Spend 5 minutes on Proof 2 from Professor Henry's note. How do you feel about it? Is it very easy to understand? How different is proof 2 from proof 1?
Explain the proof to your friend. Which one would you prefer if asked to write a proof of a non-trivial statement.\\
Spend 5 minutes on Proof 3 from Professor Henry's note. How do you feel about it? Is it very easy to understand? How different is proof 3 from the previous ones?
Which one would you prefer if asked to write a proof of a non-trivial statement.\\

\end{document}  