\documentclass[11pt, oneside]{article}   	% use "amsart" instead of "article" for AMSLaTeX format
\usepackage{geometry}                		% See geometry.pdf to learn the layout options. There are lots.
\geometry{letterpaper}                   		% ... or a4paper or a5paper or ... 
%\geometry{landscape}                		% Activate for rotated page geometry
%\usepackage[parfill]{parskip}    		% Activate to begin paragraphs with an empty line rather than an indent
\usepackage{graphicx}				% Use pdf, png, jpg, or eps§ with pdflatex; use eps in DVI mode
								% TeX will automatically convert eps --> pdf in pdflatex		
\usepackage{amssymb}
\usepackage{enumitem}
\usepackage{amsmath}
\usepackage{amsthm}
\newtheorem{theorem}{Theorem}
\newtheorem*{theorem*}{Theorem}
%SetFonts

%SetFonts


\title{CSC 341: Automata, Formal Languages, and Complexity Theory}
\author{Worksheet \#01}
\date{}							% Activate to display a given date or no date

\begin{document}
\maketitle
Name \underline{\hspace{10cm}}
\\
\section{Review}
\par Answer the following questions:
\begin{enumerate}[noitemsep, nolistsep]
\item In few sentences state the objectives of this class.
\item Name the central areas of the theory of computation. Define each area in one/two sentence(s).
\item Think of a real-life problem (A) that can be reduced to another problem (B) which you have already learnt in the some of the previous classes.
\item We have two problems at hand:
 \begin{enumerate}
 \item Sort the list of enrolled students of a class in alphabetic order of their names.
 \item From an undirected graph, find the largest clique. \emph {(Note: A clique of a graph G is a complete subgraph of G.)}
\end{enumerate}
Are both of the problems equally difficult? Why/why not? Just put down your thoughts. Your statement (at least at this moment) does not have to me mathematically proven.
\end{enumerate}
\newpage
\section{Proof Techniques}
Using either construction/contradiction/induction strategy, prove that

\begin{theorem*}
Two integers $a$ and $b$ are consecutive if and only if $b = a + 1$. You can safely assume that $b > a$.
\subsection{Solution: {\bf Proof by Contradiction}}
Let us assume that $a$ and $b$ are two consecutive integers and $b = a + m$, where $m \ne 1$. \\
Since $1$ is the smallest positive integer, we assume that $m > 1$. Now if we look at the natural number series,
 there are $(m-1)$ distinct number, $a+1, a+2, \dots , a+ (m-1)$, between $a$ and $b$.
 Hence,  we arrive at the decision that $a$ and $b$ are consecutive if and only if $m = 1$.  
\end{theorem*}

\end{document}  