%\input /home/walker/setup.standard
%\input /home/walker/setup.bigfonts
%\input /usr/share/texmf-texlive/tex/generic/misc/psfig.sty

\input /Users/walker/setup.standard
\input /Users/walker/setup.bigfonts
\input psfig.sty

\centerline {\bold The Size of the Set of Subsets, together with Alternative Proofs}
\centerline {\ital by Henry M. Walker, Grinnell College}

{\bold Theorem:}  Let $S$ be a set with $n$ elements.\hfil\break
Then $S$ has $2^n$ subsets.

\bigskip
\line{\hrulefill}
\bigskip
{\bold Proof 1:}\hrulefill

Let $P(S)$ be the set of all subsets of $S$, and let $H$ be the set of $n$-character strings of $0$'s and $1$'s.\hfil\break 
Order the elements of $S$ as $s_1 , s_2, \dots s_n$.
\medskip
Define function $f: P(S) \to H$ as follows:  

{\parskip = 0pt
\parindent = 20pt
For $Q \in P(S)$, let $f(Q)=r_1 r_2 \dots r_n$ where $r_i $ is 1 if $s_i\in Q$ and $0$ otherwise.

}
\medskip
$f$ is clearly 1-1 and onto (i.e., $f$ is a bijection), and the theorem follows.

\bigskip
\line{\hrulefill}
{\bold Proof 2:}
Order the elements of $S$ as $s_1 , s_2, \dots s_n$, and let $P(S)$ be the set of all subsets of $S$.
\hfil\break
Let $H$ represent the binary numbers between $0$ and $2^n-1$.  Since all numbers in this range may be represented by $n$ binary digits, $H$ includes $n$-digit sequences

{\parindent = 20pt 
\item {}$000\dots 00$ ($n$ $0$'s), 

\parskip = 0pt
\item {}$000 \dots 01$ ($n-1$ $0$'s followed by a $1$), 
\item {}$\dots$ ,
\item {}$111\dots 11$ ($n$ $1$'s).  

}

\medskip
Each element $Q$ in $P(S)$ is a subset of $S$ contains zero or more elements of $S$, and one can determine whether each $s_i$ is in $Q$ for $i= 1, 2, \dots n$.


Define function $f: P(S) \to H$ as follows:

{\parskip = 0pt
\parindent = 20pt
For $Q \in P(S)$, let $f(Q) = r_1 r_2 \dots r_n$, where 
$r_i$ is the digit $1$ if $s_i \in S$ and $0$ otherwise, for $i= 1, 2, \dots n$.
}

{\ital Claim:}  $f$ is 1-1 and onto (i.e., $f$ is a bijection)\hfil\break
{\ital Proof of Claim:}  \hfil\break
{\ital 1-1:}  For subsets  $X_1$ and $X_2$  of $S$, $f(X_1)$ and $f(X_2)$ will differ in each digit for which an element in one subset is not also in the other.  Thus, $X_1 \ne X_2 \Longrightarrow f(X_1) \ne f(X_2)$\hfil\break
{\ital onto:} Given an $n$-digit binary number $b$ (i.e., a number in $H$, let $X$ be the subset of $S$ which contains element $s_i$ if and only if the $i^{th}$ digit of $b$ is $1$.  Then $f(S) = b$.

Since $f$ is 1-1 and onto, each subset of $S$ is paired with a number in $H$.  Since elements of $H$ provide a count (in binary) of numbers from $0$ to $2^n -1$, $H$ has size $s^n$, and $P(S)$ also must have size $s^n$.

\vfill\eject

{\bold Proof 3:}  Let $P(S)$ be the set of all subsets of $S$.\hfil\break
We construct a mechanism (called a function) to count the elements of $P(S)$.

{\ital Step 1:} We examine the binary numbers from $0$ through $2^n-1$.
\hfil\break
{\ital Discussion of Step 1:}
In binary notation, the digits represent powers of 2.  For $k+1$ digits, the bits represent the powers $2^k, 2^{k-1}, 2^{k-2}, \dots 2^1, 2^0$.   Thus, the number 1, followed by $k$ $0$'s (i.e., $1000\dots 000$) represents the number $2^k$.  Subtracting 1 from this number in binary yields $111\dots 111$ ($k$ $1$'s) or $2^k-1$.  Turning to the theorem at hand, the number $2^n$ is represented in binary by $n$ $1$s.  

If we count in binary, therefore, the numbers $0$ through $2^n-1$ may be represented as 0, 1, 10, 11, \dots , $n$ $1$s.  If we add leading $0$s to these numbers as needed, so that each number from $0$ through $2^n-1$ is written using $n$ bits, the resulting sequence becomes:

{\parindent = 20pt 
\item {}$000\dots 00$ ($n$ $0$'s), 

\parskip = 0pt
\item {}$000 \dots 01$ ($n-1$ $0$'s followed by a $1$), 
\item {}$000 \dots 10$ ($n-2$ $0$'s followed by a $10$), 
\item {}$000 \dots 11$ ($n-2$ $0$'s followed by a $11$), 
\item {}$\dots$ ,
\item {}$111\dots 11$ ($n$ $1$'s).  

}

For future reference, define the set $H$ to be this collection of binary numbers between $0$ and $2^n$.


{\ital Step 2:} We consider a representation of the elements of $S$.
\hfil\break
{\ital Discussion of Step 2:}
Since $S$ is a given set of $n$ elements, we may fix an order for these elements, and then label the elements as the sequence $s_1, s_2, \dots s_n$ 

{\ital Step 3:} We develop a mechanism to count all subsets of $P(S)$.\hfil\break
{\ital Discussion of Step 3:}
We define a function $g: H \to P(S)$ as a mechanism to count all elements in $P(S)$.

Let $b$ be a binary integer between $0$ and $2^n$., and let $b_1 b_2 \dots b_n$ be its binary expansion in the set $H$.  

Define function $g: h \to P(S)$ by $g(b) = \{s_i \mid b_i = 1\}$ for $i=1, \dots n$.

With this definition, $g$ is well defined, since each bit in a binary integer $b$ corresponds unambiguously to an element of $S$, and reading along the bits of $b$ indicates exactly what subset will correspond to $g(b)$.

To show $g$ is 1-1, consider two binary numbers $h_1$ and $h_2$ in $H$, and suppose $g(h_1) = g(h_2)$.  Let $T = g(h_1) = g(h_2)$,
For each $i$ between $1$ and $n$, 

{\parindent = 20pt
\item {} if $s_i \in T$, then the $i^{th}$ bit of both $h_1$ and $h_2$ must be $1$, by the definition of $g$.

\parskip = 0pt
\item {} if $s_i \not\in T$, then the $i^{th}$ bit of both $h_1$ and $h_2$ must be $0$, by the definition of $g$.

}

Putting these bits together, $g(h_1) = g(h_2)$ requires that every bit of $h_1$ is the same as the corresponding bit of $h_2$, and it follows that $h_1=h_2$.

To show $g$  is onto, consider a subset $R$ of $S$.  From $R$, construct a binary integer $b$ with bits $b_1 b_2 \dots b_n$ as follows:

{\parindent = 20pt

\item {}  For $i = 1$ to $n$, let $b_i = 1$ if $s_i \in R$ and let $b_i = 0$ otherwise.

}

By the construction and definition of $g$, $g(b) = R$, so $g$ is 1-1.

{\ital Step 4:}  The Theorem follows by counting.\hfil\break
{\ital Discussion of Step 4:}
Altogether, function $g$ provides a 1-1 correspondence between the numbers $0$ and $2^n-1$, effectively providing a mechanism that uses these integers to count each subset of $S$ exactly once.  

\vfill\eject

{\bold Proof 4:}  Suppose set $S$ has $n$ elements.\hfil\break
If $n=0$, then $S$ is the empty set, and its only subset is itself.

If $n>0$, pick an element $s \in S$, and let $U$ be the set $S$ with the element $s$ removed. Since $U$ has $n-1$ elements, the power set $P(U)$ of $U$ contains $2^{n-1}$ subsets. Also, let $P^*(U)$ consist of all subsets in $P(U)$ with the element $S$ added.  

Since $P(S) = P(U)  \bigcup P^*(U)$,  $P(U)$ and $P^*(U)$ are disjoint, and $P(U)$ and $P^*(U)$ each have $2^{n-1}$ elements, it follows that $P(S)$ has $2^n$ elements. 

\bigskip
\line{\hrulefill}
{\bold Proof 5:}  Suppose set $S$ has $n$ elements.\hfil\break
The proof proceeds by mathematical induction on $n$ with the following induction hypothesis:

{\parindent = 20pt  

\item {} $IH(n)$:  If $S$ is any set with $n$ elements, then it has exactly $2^n$ subsets.

}

{\ital Base case ($n=0$):}  If $n=0$, then $S$ is the empty set.  The only subset of the empty set is the empty set itself, so there are exactly $1=2^0$ subsets, as required by $IH(0)$.

{\ital Induction case ($n>0$):}  Assume the Induction Hypothesis $IH(k)$ for integers $k < n$; the following argument shows that $IH(n)$ is true as well.

Since $n>0$, the set $S$ has at least one element.  Pick $s$ as one such element, and consider the set $U$ obtained by removing the element $s$ from $S$, sometimes written $U = S - \{s\}$.  

Since one element has been removed from $S$, $U$ has $n-1$ elements, the Induction Hypothesis $IH(n-1)$ applies to $U$, and $U$ has $2^{n-1}$ subsets.  Label this collection of $2^{n-1}$ subsets as $W$.

Next, form a new collection $N$ of sets by adding the element $s$ to each subset in $W$.  Since each element of $W$ is a subset of $S$ and since $s$ is an element of $S$, each element of $N$ is also a subset of $S$.

Now, suppose $A$ and $B$ are two distinct elements of $W$; that is, $A$ and $B$ are distinct subsets of $U = S - \{s\}$.  Since $A$ and $B$ are distinct, there is at least one element in $A$ that is not in $B$ or one element in $B$ that is not in $A$.  That is, $A$ and $B$ differ by some element $q \in U$.  Since neither $A$ or $B$ contain $s$, $q \ne s$, so $q$ remains a difference between $A \bigcup \{s\}$ and $B \bigcup \{s\}$.  Altogether, this shows that the number of elements in $N$ is the same as the number of elements in $W$, namely $2^{n-1}$.

In addition, no element in $W$ is also in $N$, since all elements in $W$ do not contain $s$, while all elements of do contain $s$.  As $W$ and $N$ are disjoint, the number of elements in $W \bigcup N$ is $2^{n-1} + 2^{n-1} = 2^n$.  Since all elements of $W \bigcup N$ are subsets of $S$, the number of subsets of $S$ must be at least $2^n$.

Finally, every subset $V$ of $S$ either contains $s$ or it does not.  

{\parindent = 20pt

\item {} If V does not contain $s$, then $V \in W \in W \bigcup N$.

\parskip = 0pt

\item {} If V does contain $s$, then $V-\{s\}$ does not contain $s$ and thus is contained in $W$.  Adding $s$ to $V-\{s\}$ places the result $N$.  Thus, $V \in N  \in W \bigcup N$.

}

Since every subset $V$ of $S$ is contained in $W \bigcup N$, the number of such subsets cannot be bigger than the size of $W \bigcup N$, which is $2^n$.  

Put together, $W \bigcup N$ contains exactly all subsets of $S$, proving $IH(n)$, which states that the number of such subsets is $2^n$.

\vfill\eject
{\bold Proof 6:}  This argument proceeds by contradiction:\hfil\break
Let $S$ be a set of $n$ elements, and suppose that the number of subsets of $S$ is not $2^n$.  Then either the number of subsets is less than $2^n$ or greater than $2^n$.  What follows examines each of these possibilities in detail.

{\ital Part 1:  The number of subsets of $S$ cannot be less than $2^n$.}

Let $P(S)$ be the collection of all subsets of $S$, and \hfil\break
let $St$ consist of all strings from the alphabet $\{0, 1\}$ of length $n$.\hfil\break  
Also, order the sets of $S$ to yield a sequence $s_1, s_2, \dots s_n$.

Next, construct a function $f: P(S) \to St$ as follows.

For a subset $Q$ of $S$, define $f(Q) = t_1 t_2 \dots t_n$, where, for each $i$,
$t_i = 1$ if $s_i \in Q$ and $t_i = 0$ if $s_i \not\in Q$.  That is, the digits of $f(Q)$ indicate whether or not element $s_i$ is in $Q$.

{\ital Claim:  $f$ is onto:}

Let $t = t_1 t_2 \dots t_n$ be any string of length $n$ over the alphabet $\{0, 1\}$; that is, let $t$ be any element in $St$.  From this string, form a set $Q$ from elements of $S$, according to the following rules:

For each $i$ between $1$ and $n$, 

{\parindent = 20pt
\parskip = 0pt
\item {} if $t_i$ is $1$, then place $s_i$ in $Q$, but 
\item {} if $t_i$ is $0$, then do not place $s_i$ in $Q$.

}

By construction, $f(Q) = t$, so $f$ is onto.

{\ital Claim:} $St$ contains $2^n$ elements.

In considering possible strings in $St$, 

{\parindent = 20pt
\item {} there are 2 choices ($0$ or $1$) for $t_1$
\parskip = 0pt

\item {} there are 2 choices for $t_2$
\item {} \dots
\item {} there are 2 choices for $t_n$

}

Choices for each digit are independent, so overall there are 
$2 \times 2 \times 2 \dots  \times 2 = 2^n$ possible strings in $St$.

Since $f$ is an onto function, and the range $St$ has $2^n$ elements, the domain of $f$ must have at least $2^n$, proving the claim for Part 1.


{\ital Part 2:  The number of subsets of $S$ cannot be greater than $2^n$.}

As in Part 1, Let $P(S)$ be the collection of all subsets of $S$, and \hfil\break
order the sets of $S$ to yield a sequence $s_1, s_2, \dots s_n$.

Also, consider all integers between $0$ and $2^n-1$ (inclusive) as represented using binary numbers.  Such numbers can be written using no more than $n$ binary digits.  However, in the case that the binary representation does not require $n$, add leading $0$'s so that all integers from $0$ through $2^n-1$ are represented as n-digit binary numbers.  For reference, label this collection of binary numbers as BN.

Now, define a function $g: BN \to P(S)$ as follows.

Let $b_1 b_2 \dots b_n$ be an n-digit binary number in $BN$.\hfil\break
Then $g(b_1 b_2 \dots b_n$) is defined as the set $Y$, where the subset $Y$ is prescribed by the rules:

{\parindent = 20pt
\parskip = 0pt
\item {} if $b_i$ is $1$, then place $s_i$ in $Y$, but 
\item {} if $b_i$ is $0$, then do not place $s_i$ in $Y$.

}

{\ital Claim:  Function $g$ is onto}

Let $Q$ be a subset of $S$.  Consider the $n$-digit binary number $b_1 b_2 \dots b_n$ constructed as follows:

{\parindent = 20pt
\parskip = 0pt
\item {} if $s_i \in Q$, set $b_i = 1$ 
\item {} if $s_i \not\in Q$, set $b_i = 0$

}

By construction, $g(b_1 b_2 \dots b_n) = Q$, showing that $g$ is onto.

Finally, since $g$ maps all integers from $0$ to $2^n-1$ onto $P(S)$, the number of elements in $P(S)$ cannot be greater than the number of integers from $0$ to $2^n-1$, namely $2^n$, proving Part 2.

\bigskip
\line {\hrulefill}

\copyright~2018 by Henry M. Walker\hfil\break
This material is distributed under a Creative Commons Attribution-NonCommercial-ShareAlike 4.0\break International license.  For details, see \hfil\break
{\programfont http://creativecommons.org/licenses/by-nc-sa/4.0/} 
 

\end
